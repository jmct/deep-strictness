\documentclass{article}

\usepackage[margin=1in]{geometry}
\usepackage{syntax}
\usepackage{float}
\usepackage{amsmath}
\usepackage{tikz}
\usetikzlibrary{arrows,automata}
\usepackage{minted}

\title{Deep Strictness: Milestone 1}
\author{Kenny, Hengchu}

\begin{document}
\maketitle

We aim to extend QuickCheck with the ability to probabilistically
verify the laziness/strictness behavior of Haskell functions. That is,
we will allow users to write a specification of the strictness
behavior of a particular function, and by fuzzing inputs to the
function, determine if the function is \emph{exactly} as strict as the
user has specified.

Before describing in detail how our proposed tool will work, we give
several definitions regarding laziness in Haskell:

A lazy function takes an \emph{input} to an \emph{output}. For now, we
consider only inputs and outputs restricted to single algebraic
datatypes which do not contain functions (that is, all functions we
can speak about are unary first-order functions). We plan to extend
this to multi-argument and higher-order functions in later stages of
this project.

When a lazy function is applied to an input, it is done so in a
\emph{context} which makes use of some part of its output---that is,
the context may not use the entire output data structure to continue
its own computation. We call the portion of the input which is used by
the context the \emph{demand} on the output. The semantics of lazy
evaluation dictate that any portion of the input which is not required
to compute the demanded portion of the output are not evaluated in the
first place---that is, evaluation only occurs as much as is necessary
to satisfy the demand of the calling context.

For example, let us consider a function \verb|f :: Int -> Int -> Int| where
\verb|f x n = fst (x, fib n)|. Although computing the second component of
the pair in \verb|f 1 100000000| would be very expensive, this computation
never occurs because the demand placed upon the pair (extracting the
first component) only requires the first component to be evaluated. We
say that \verb|f| is \emph{strict} in its first parameter (\verb|x|) and
\emph{lazy} in its second parameter (\verb|n|).

We can capture this notion formally as a function from a given demand
of the output to a demand of the input. We call this a \emph{demand
  specification}. Here, we can concretely represent a demand as a
\emph{subshape} of the input/output---that is, a data structure
corresponding to one of these data types, but which can be truncated
at any arbitrary position in the structure of the data type. We call
this derived data type the \emph{demand type} corresponding to a data
type.

For example, the corresponding demand type for the type \verb|(Int, Int)| would be:

\verb|Demand (Demand Int, Demand Int)|

where

\begin{minted}{haskell}
data Demand a = ~   -- ^ Not demanding
              | ! a -- ^ Demanding the constructor at this level
              | *   -- ^ Demanding a primitive or nullary constructor
\end{minted}

All the values of these types are:
\verb|~, (~, ~), ! (*, ~), ! (~, *)|, and \verb|! (*, *)|. These
represent all of the possible demands which a context could place on
the type \verb|(Int, Int)|.

Suppose we have a function \verb|g :: (Int, Int) -> Int|. The corresponding
demand specification for g would be a function
\verb|gSpec :: Demand Int -> Demand (Demand Int, Demand Int)|.

Concretely, consider the function \verb|fst| (for integers):

\begin{minted}{haskell}
fst :: (Int, Int) -> Int
fst (a, b) = a
\end{minted}

An accurate demand specification for \verb|fst| is:

\begin{minted}{haskell}
fstSpec :: Demand Int -> Demand (Demand Int, Demand Int)
fstSpec ~ = ~
fstSpec * = ! (*, ~)
\end{minted}

This states that if we demand nothing from the result of \verb|fst|, we do
not need to evaluate its argument at all. (This is true for
\emph{every} function in a lazy language!) However, if we demand the
resultant integer from \verb|fst|, we must evaluate the first integer in
the input pair, but we do not need to evaluate the second element of
the pair.

For functions like \verb|fst|, the demand on input is a \emph{static
  demand}: that is, for each demand on the output, any input to the
function will be evaluated to the same degree.

This is not true in general, and not merely in pathological
examples. Many (perhaps most!) useful functions have dynamic demand
behavior---that is, their demand specification depends on the values
of the input. We say that such functions have \emph{dependent demand}:
that is, the demand specification of these functions must consider the
input values of the function as well.

\begin{minted}{haskell}
take :: (Int, [Int]) -> [Int]
take (n, xs) =
  case (n, xs) of
    (0, _)     -> []
    (n',x:xs') -> x:(take ((n'-1), xs'))
    (n', [])   -> []
\end{minted}

Just given some demand on the output list, we can't guess the demand
on the second element of the input pair, although we do know that the
first element of the pair will always be evaluated.

Suppose we have demand the first 2 elements of some result of take,
there is no one input demand that corresponds to this output
demand. However, if we know what the first argument to \verb|take| is, we
can determine the degree to which the input list must be
evaluated. Note that the demand on the output still is another factor
in determining the demand on the input, since if we demand nothing on
the output, then nothing on the input will be demanded.

The demand specification of \verb|take| could be

\begin{minted}{haskell}
takeSpec :: (Int, [Int]) -> Demand [Demand Int] -> Demand (Demand Int, Demand [Demand Int])
takeSpec _       ~    =  ~
takeSpec (0, _)  ![]  =  !(*, ~)
takeSpec (n, xs) !ds  =
  case (ds, xs) of
    ([],     _)    -> !(*, ![])
    (d:ds', x:xs') ->
      let !(_, !ds'') = takeSpec (n-1, xs') !ds' in
      !(*, !(d:ds''))
\end{minted}

Determining statically whether a function meets its demand spec is
undecidable due to the Halting Problem. So, instead of statically
approximating these behavior, we choose to apply runtime
instrumentation with randomized testing, this gives a high level of
confidence.

We choose to implement this as an extension to QuickCheck
(CITATION). We will use generic programming to derive the demand types
of functions under consideration. We will use existing QuickCheck
generator infrastructure to fuzz inputs to the function, and demands
on the output of the function. Because only certain shapes of demand
are realizable on a given output, we specialize the demand generators
to produce only sensible demands on each invocation of the
function. We then run the function on the random input to obtain
un-evaluated output, and place a random demand on this output while
instrumenting the function to determine the resultant demand on the
input. We can then use the demand specification to check whether the
demand specification exactly matches what the actual demand behavior
of the function was in this trial.

Because we integrate with the existing functionality of QuickCheck, we
can employ its \emph{shrinking} abilities to produce minimal
counterexamples of input/demand pairs which fail to satisfy a given
demand specification, if this specification is faulty.

Note that there are two ways in which a specification may be faulty:
it may fail to produce an input demand which exactly matches the real
observed input demand, or it may fail to cover all possible realizable
combinations of input and demand.

To avoid unnecessary metaprogramming, we choose not to calculate an
instrumented version of the function under consideration. We do not
need to resort to such strategies because we can instead instrument
the input data structure to report to us how it is evaluated by the
function. As such, we may treat our functions as a black box.

In order to instrument a lazy data structure, we must use so-called
``unsafe'' features of Haskell---in particular, the function
\verb|unsafePerformIO| which allows us to attach a callback to a particular
lazy value so that the callback is run when that value is
evaluated. We may traverse the generated input data structure to add
such a callback to every node in that structure. By tracking which of
these callbacks are invoked when we demand a particular part of the
output of some function applied to the instrumented structure, we
produce an exact representation of the input demand of this function
given a particular output demand and input value.

Once we have obtained a pair of generated output demand and observed
input demand, it's trivial to run the demand specification on the
output demand and check whether it exactly matches the observed
demand. If it does not, we may repeat this process by shrinking inputs
to the function to compute a minimal example exhibiting the detected
violation of the demand specification.

We need not only to consider how to \emph{check} such specifications,
but also how to specify them in the first place! To that end, we will
use Haskell's generic programming features to provide a syntax for
users to succinctly write these demand specifications without worrying
about the details of our implementation. Ideally, this syntax would
look very similar to the syntax we use above, but we may need to make
some concessions due to the limitations of the implementation strategy
we choose for this interface.

Moving forward, we want to allow users to check demand specifications
of multi-argument functions, higher-order functions, and functions on
data structures which themselves contain functions.

\end{document}
